\documentclass{report}
\usepackage[utf8]{inputenc}
\usepackage{geometry}
\newgeometry{left=3cm,bottom=2cm, right=3cm, top=2cm}
\usepackage{graphicx}
\usepackage{framed}
\usepackage{parskip}
\usepackage{amssymb}
\usepackage{amsmath}
\usepackage{amsthm}
\usepackage[dvipsnames]{xcolor}
\usepackage{tcolorbox}
\usepackage{charter}

\newenvironment{blackbox}[1][Black]
  {\begin{tcolorbox}[colframe=#1,colback=white]}
  {\end{tcolorbox}}
%-------------------------------------------------------------------------


\title{\huge{\textit{\textbf{Accounting and Finance}}}\\
\Large\textbf{\textit{Complete Summarised Notes}}\\
BUS021 - Year 1 Semester 1}

\author{Adam Curry}
\date{September - December 2022}

\begin{document}
\maketitle
\tableofcontents

%-------------------------------------------------------------------------
\chapter{Introduction to Accounting}
%-------------------------------------------------------------------------
\begin{blackbox}
    \textbf{\textit{Learning Outcomes:}}\\
    \begin{itemize}
        \item Explain the nature and roles of accounting and finance
        \item Identify the main users of financial information and discuss their needs
        \item Identify and discuss the characteristics that make accounting information useful
        \item Explain the purpose of a business and describe how businesses are organised and structured
    \end{itemize}
\end{blackbox}

%-------------------------------------------------------------------------
\section{Introduction}

Accounting is concerned with collecting, analysing, and communicating financial information. The ultimate aim is to help those using this information to make more informed decisions.\\

Finance (or financial management), like accounting, exists to help decision-makers. It is concerned with the ways in which funds for a business are raised and invested. In essence, a business exists to raise funds from investors (owners and lenders), and then use those funds to make investments (in equipment, premises, inventories and so on) in order to create wealth. As businesses often raise and invest large amounts over long periods, the quality of the financial and investment decisions can profoundly impact their fortunes.\\
An understanding of finance should help identify:
\begin{itemize}
    \item the main forms of finance available
    \item the costs, benefits, and risks of each form of finance
    \item the risks associated with each form of finance
    \item the role of financial markets in supplying finance
\end{itemize}

%-------------------------------------------------------------------------
\section{Main Users of Financial Information}

\begin{itemize}
    \item Owners - whether to invest more into the business or to sell all or part of the investment currently held.
    \item Managers - whether the performance of the business needs to be improved. Performance to date would be compared with some ‘benchmark’ to decide whether action needs to be taken. Managers may wish to consider a change in the company’s future direction.
    \item Lenders - whether to lend money to the company.
    \item Suppliers - whether to continue to supply the company with goods and if so whether to supply these on credit.
    \item Investment Analysts - whether to advise clients to invest in the company.
    \item Community representatives - whether to allow the company to expand its premises, and whether to provide economic support.
    \item Government - whether the company should pay higher tax, whether financial support is needed.
    \item Employees and their representatives - whether to continue working for the company and whether to demand higher rewards.
    \item Competitors - how best to compete against the company, if it is possible to compete profitably with the company.
    \item Customers - may assess the company’s ability to continue in business and meet customers’ needs.\\
\end{itemize}

One way of viewing accounting is as a form of service. The user groups identified above can be seen as ‘clients’ and the accounting information produced can be seen as the service provided. Another way is as an information system.\\

\textit{\textbf{The Accounting Information System:}}
\begin{itemize}
    \item Information identification
    \item Information recording
    \item Information analysis
    \item Information reporting
\end{itemize}

%-------------------------------------------------------------------------
\section{Usefulness of Accounting Information}

It is normally very difficult to assess the impact of accounting on decision-making. One situation arises, however, where the impact of accounting information can be observed and measured. This is where the shares (portions of ownership of a business) are traded on a stock exchange. The evidence shows that, when a business makes an announcement concerning its accounting profits, the price at which shares are traded at and the volume of the shares traded often change significantly.

To be useful to users, particularly investors and lenders, the information provided should possess certain qualities. It must be relevant and faithfully represented.

\begin{blackbox}
    \textbf{Relevance:}
    \begin{itemize}
        \item It must help predict future events, help confirm past events, or both.
        \item By confirming past events, users can check the accuracy of their earlier predictions.
        \item In order to be relevant, the information must cross a threshold of materiality. An item of information should be considered material (or significant) if its omission would change the decisions that users make.
    \end{itemize}
\end{blackbox}

\begin{blackbox}
    \textbf{Faithful Representation:}
    \begin{itemize}
        \item Accounting information should portray what it's supposed to portray.
        \item To provide a perfectly faithful portrayal, the information provided should be complete, incorporating everything needed to understand what is being portrayed.
        \item It should be free from error. This is not to say it must be perfectly accurate, as this may mot be possible, since accounting information often contains estimates, such as future costs and sales, which may turn out to be inaccurate. Nevertheless, estimates can still be faithfully portrayed providing they are accurately described and properly prepared.
    \end{itemize}
\end{blackbox}

 Where accounting information is both relevant and faithfully represented, there are other qualities that, if present, can enhance its usefulness.
 \begin{blackbox}
     \textbf{Comparability:}\\
     When making choices, users of accounting information often seek to make comparisons. They may want to compare the performance of the business over time (e.g. profit this year compared to last). They may want to compare aspects of business performance (e.g. the level of sales achieved during the year) to those of similar businesses.
 \end{blackbox}

 \begin{blackbox}
     \textbf{Verifiability:}\\
     Accounting information is verifiable where different, independent experts could reach a broad agreement that it provides a faithful portrayal. Verification can be direct, such as checking a bank account balance, or indirect, such as checking the underlying assumptions and methods used to derive an estimate of a future cost.
 \end{blackbox}

 \begin{blackbox}
     \textbf{Timeliness:}\\
     Accounting information should be made available in time for users to make their decisions.
 \end{blackbox}

 \begin{blackbox}
     \textbf{Understandability:}\\
     Accounting information should be set out in as clear and concise form as possible. Some information may be too complex to be presented in an easily digestible form, however, this should still not be ignored, as it would only give a partial view of financial matters.
 \end{blackbox}
%-------------------------------------------------------------------------
\section{Management and Financial Accounting}

Accounting is usually seen as having two distinct strands:
\begin{itemize}
    \item management accounting, which seeks to meet the accounting needs of managers
    \item financial accounting, which seeks to meet the needs of owners and lenders (and other users)\\
\end{itemize}

\begin{blackbox}
    \textbf{Nature of the report produced:}\\
    Financial accounting reports tend to be general purpose. They are aimed primarily at providers of finance but contain information that should be useful for a broad range of external users.\\
    Management accounting reports are often specific-purpose reports, designed with a particular decision in mind and/or for a particular manager.
\end{blackbox}

\begin{blackbox}
    \textbf{Level of detail:}\\
    Financial accounting reports provide users with a broad overview of the performance and position of the business for a period, so information is aggregated and detail is often lost.\\
    Management accounting reports often provide managers with considerable detail to help them with a particular operational decision.
\end{blackbox}

\begin{blackbox}
    \textbf{Regulations:}\\
    Financial accounting reports, for many businesses, are subject to accounting regulations imposed by the law and accounting rule makers. These regulations often require a standard content and format to be adopted.\\
    Management accounting reports are not subject to regulation and can be designed to meet the needs of particular managers.
\end{blackbox}

\begin{blackbox}
    \textbf{Reporting interval:}\\
    Most businesses produce financial reports on an annual basis (though some produce half-yearly and quarterly reports).\\
    Management accounting reports will be produced as frequently as needed by managers. A sales manager may require routine sales reports on a daily, weekly, or monthly basis, to closely monitor performance. Special purpose reports may also be prepared, for example, where an evaluation is required of a proposed investment in new equipment.
\end{blackbox}

\begin{blackbox}
    \textbf{Time orientation:}\\
    Financial accounting reports reflect the performance and position of the business for the past period.\\
    Management accounting reports often provide information concerning future performance as well as past performance.\\
    It is an oversimplification to suggest that financial accounting reports never incorporate expectations concerning the future. Occasionally, businesses will release projected information to other users in an attempt to raise capital or fight off unwanted takeover bids. Even the preparation of routine financial accounting reports typically requires making some judgements about the future.
\end{blackbox}

\begin{blackbox}
    \textbf{Range and quality of information:}\\
    Financial accounting reports concentrate on information that can be quantified in monetary terms.\\
    Management accounting also produces such reports, but is also more likely to produce reports that contain information of a non-financial nature, such as physical volume of inventories, number of sales orders recieved, number of new products launched, and so on. 
\end{blackbox}

%-------------------------------------------------------------------------
\section{Types of Business Organisation}

\subsection{Sole Proprietorship}
\begin{itemize}
    \item A Sole Proprietorship is where an individual is the sole owner.
    \item This type of business is very easy to set up and has no formal procedures, and operations can commence immediately.
    \item A Sole Proprietorship has unlimited liability, meaning no distinction is made between the proprietor's personal wealth and that of the business.
\end{itemize}

\subsection{Partnership}
\begin{itemize}
    \item A Partnership exists when two or more individuals start a business, and they have much in common with Sole Proprietorships.
    \item They too are easy to set up, and the partners tend to have unlimited liability.
\end{itemize}

\subsection{Limited Company}
\begin{itemize}
    \item The liability of owners is 'limited', as individuals investing in company shares are liable for only debts incurred by the company up to the amount they have invested.
    \item To create a limited company, documents of incorporation must be prepared. Furthermore, a framework of regulations exists that places obligations on limited companies concerning the way they conduct their affairs.
    \item Regulations include annual financial reports must be produced and made available to owners and lenders, and these reports must be lodged with the Registrar of Companies. With the exception of small companies, these financial reports must be audited.\\
\end{itemize}

A major disadvantage of partnerships compared to limited companies is that it is not normally possible to limit the liability of the partners. However, there is a hybrid form of ownership - a Limited Liability Partnership (LLP). This has many of the attributes of a normal partnership, but the LLP rather than the individual partners is responsible for debts incurred. This is often used by accountants and solicitors.

%-------------------------------------------------------------------------
\chapter{Statement of Financial Position}
%-------------------------------------------------------------------------
\begin{blackbox}
    \textbf{\textit{Learning Outcomes:}}\\
    \begin{itemize}
    \item Explain the nature and purpose of the statement of financial position.
    \item Prepare and interpret a statement of financial position.
    \item Discuss accounting conventions underpinning the statement of financial position, and its uses and limitations.
\end{itemize}
\end{blackbox}

%-------------------------------------------------------------------------
\section{Assets}

\begin{blackbox}
    \textbf{Asset:}\\
    An asset is a resource held by a business.
    \begin{itemize}
        \item It must be an economic resource - providing potential economic benefits - but these benefits must not equally be available to others.
        \item The economic resource must be under the control of the business - giving the business the exclusive right to decide how the resource is used and the right to any benefits from it.
        \item The economic resource must be capable of measurement in monetary terms - an estimate with great uncertainty is fine, as it can still be reported as an asset for inclusion in the statement of financial position.
    \end{itemize}
\end{blackbox}

Examples of assets include;
\begin{itemize}
    \item Property - land and buildings
    \item Plant and equipment
    \item Fixtures and fittings
    \item Patents and Trademarks
    \item Trade receivables
    \item Investments\\
\end{itemize}

\textbf{Tangible assets} are assets that have a physical substance (eg inventories).\\
\textbf{Intangible assets} are assets that have no physical substance but provide future benefits (eg patents).

\begin{blackbox}
    \textbf{Current Assets:}\\
    Current assets are assets held for the short term;
    \begin{itemize}
        \item Those held for sale or consumption during the business' normal operating cycle.
        \item Those expected to be sold within a year.
        \item Those held principally for trading.
        \item Those that are cash or near cash (like short-term investments).
    \end{itemize}
\end{blackbox}

The operating cycle is the time between buying a product or service and receiving the cash on its sale.\\
Common current assets include inventories, trade receivables, and cash.

\begin{blackbox}
    \textbf{Non-Current Assets:}\\
    Non-current (fixed) assets are assets that simply do not meet the definition of current assets. They tend to be held for long-term operations.
\end{blackbox}

Non-current assets include property, plant and equipment.
%-------------------------------------------------------------------------
\section{Claims}

A claim is an obligation of the business to provide cash, or another benefit to an outside party.

\begin{blackbox}
    \textbf{Equity:}
    \begin{itemize}
        \item Equity represents the claim of the owners against the business, and is sometimes referred to as owner's capital.
        \item In accounting, a clear distinction is made between the business and the owners. 
        \item Therefore, funds contributed by the owner will be seen as coming from outside the business will claim against the business in its statement of financial position.
    \end{itemize}
\end{blackbox}

\begin{blackbox}
    \textbf{Liabilities:}
    \begin{itemize}
        \item Liabilities represent the claims of other parties, apart from the owners.
        \item They involve an obligation to transfer economic resources as a result of past transactions or events. A liability incurred by a business cannot be avoided and so will remain a liability until it is settled.
    \end{itemize}
\end{blackbox}

\begin{equation}
    Assets = Equity + Liabilities
\end{equation}

\begin{blackbox}
    \textbf{Current Liabilities:}\\
    Current liabilities are amounts due for settlement in the short term;
    \begin{itemize}
        \item They are expected to be settled within the business' normal operating cycle.
        \item They exist principally as a result of trading.
        \item They are due to be settled within a year after the date of the relevant statement of financial position.
        \item There is no right to defer settlement beyond a year after the date of the relevant statement of financial position. 
    \end{itemize}
\end{blackbox}

\begin{blackbox}
    \textbf{Non-Current Liabilities:}\\
    Non-current liabilities represent amounts die that do not meet the definition of current liabilities and so represent longer-term ones.
\end{blackbox}

%-------------------------------------------------------------------------
\section{Accounting Conventions}

\begin{blackbox}
    \textbf{Business Entity Convention:}\\
    For accounting purposes, the business and its owners are treated as distinct and separate.\\
    The business entity convention must be distinguished from the legal position that may exist between businesses and their owners.\\
    The business entity convention applies to all businesses.
\end{blackbox}

\begin{blackbox}
    \textbf{Historic Cost Convention:}\\
    The historic cost convention holds that the value of assets shown on the statement of financial position should be based on their historic cost (aquisition cost).
\end{blackbox}

\begin{blackbox}
    \textbf{Prudence Convention:}\\
    The prudence convention holds that caution should be exercised when preparing financial statements.\\
    Its ultimate purpose is to avoid the risk that the financial strength of a business will be overstated, thereby resulting in poor user decisions.
\end{blackbox}

\begin{blackbox}
    \textbf{Going Concern Convention:}\\
    The going concern convention states financial statements should be prepared on the assumption a business will continue operations for the foreseeable future unless there is evidence to the contrary.
\end{blackbox}

\begin{blackbox}
    \textbf{Dual Aspect Convention:}\\
    The dual aspect convention states that each transaction has two aspects, both affecting the statement of financial position.\\
    This means that, for example, the purchase of a computer for cash results in an increase in one asset (computer), and a decrease in the other (cash).
\end{blackbox}

%-------------------------------------------------------------------------
\section{Valuing Assets}

\begin{blackbox}
    \textbf{Goodwill:}\\
    The term 'goodwill' is used to cover various attributes such as the quality of the products, the skill of employees, and the relationship with customers.\\
    The term 'product brands' is also used to cover various attributes, such as the brand image, the quality of the product, the trademark, etc.\\
    When these have been generated internally, it is hard to determine their cost, etc, so are excluded from the statement of financial position.\\
    When goodwill is acquired, some form of valuation must take place and a price will be agreed, and it then can be included as an asset.
\end{blackbox}
\pagebreak

Non-current assets have useful lives that are either finite or indefinite.\\

Benefits from assets with finite useful lives will be used up over time as a result of market changes, wear and tear, etc. The amount used up - \textbf{depreciation} - must be measured for each reporting period for which the assets are held.\\
The total depreciation that has accumulated must be deducted from its cost. This net figure is referred to as the \textbf{carrying amount.} \\

Benefits from assets with indefinite lives may or may not be used up over time. Property is usually an example of a tangible non-current asset with an indefinite life. Purchased goodwill could be an example of an intangible one, but this is not always the case.\\
These assets are not subject to routine depreciation each reporting period.\\

\begin{blackbox}
    \textbf{Fair Values:}\\
    Non-current assets may be recorded using fair values provided that these values can be measured with a fair degree of certainty.\\
    These values are market-based, and represent the selling price that can be obtained for the asset.
\end{blackbox}

\begin{blackbox}
    \textbf{Impairment Loss:}\\
    All non-current assets are at risk of suffering a significant fall in value. This may be caused by changes in the market, technological obsolescence, etc. When this occurs, the asset value is said to be impaired. The amount by which the asset value is reduced is known as impairment loss.
\end{blackbox}

%-------------------------------------------------------------------------
\section{Statement of Financial Position}

\begin{blackbox}
The statement of financial position is the oldest of the three main financial statements and helps users in the following ways:
\begin{itemize}
    \item It provides insights about how the business is financed and how its funds are deployed.
    \item It can provide a basis for assessing the value of the business.
    \item Relationships between assets and claims can be assessed. 
    \item Performance can be assessed.
\end{itemize}
\end{blackbox}

%-------------------------------------------------------------------------
\chapter{The Income Statement}
%-------------------------------------------------------------------------
\begin{blackbox}
    \textbf{\textit{Learning Outcomes:}}\\
    \begin{itemize}
    \item Explain the nature and purpose of the income statement.
    \item Prepare and interpret an income statement.
    \item Discuss accounting conventions underpinning the income statement, and its uses and limitations.
\end{itemize}
\end{blackbox}

%-------------------------------------------------------------------------
\section{The Income Statement}

Businesses exist to generate wealth, or \textbf{profit}. The income statement, or profit and loss account, measures and reports how much profit a business has generated over a period.\\

To measure profit for a particular period, the total revenue generated must be identified, as must the total expenses.\\

Examples of revenue include;
\begin{itemize}
    \item Sales of goods
    \item Fees for services
    \item Subscriptions
    \item Interest received\\
\end{itemize}

Examples of expenses include;
\begin{itemize}
    \item The cost of sales
    \item Salaries and wages
    \item Rent
    \item Insurance\\
\end{itemize}

The income statement simply shows the total revenue generated during a particular reporting period and deducts from this the total expenses incurred in generating that revenue. 

\begin{equation}
    Profit = Total \: Revenue - Total \: Expenses \: Incurred
\end{equation}

\begin{equation}
    Assets = Equity + Profit + Liabilities
\end{equation}

\begin{blackbox}
    \textbf{Gross Profit:}\\
    Gross profit represents the profit from buying and selling goods, without taking into account any other revenues or expenses.
\end{blackbox}

\begin{blackbox}
    \textbf{Operating Profit:}\\
    Operating expenses (overheads) incurred in running the business (salaries, rent, etc) are deducted from the gross profit and result in the operating profit.
\end{blackbox}

\begin{blackbox}
    \textbf{Profit for the Period:}\\
    After establishing the operating profit, any non-operating income, such as interest receivables which is added, and interest payables which is deducted.\\
    This final measure of wealth generated represents the amount attributable to the owners and will be added to the equity figure in the statement of financial position.
\end{blackbox}

%-------------------------------------------------------------------------
\section{Recognising Revenue}

The amount a business is entitled for providing goods or services to a customer should be recognised as revenue, as soon as control of the goods or services is transferred to the customer, as at this point the business has satisfied its obligations toward the customer.\\

To determine when control has passed;
\begin{itemize}
    \item Physical possession passes to the customer.
    \item The business has the right to demand payment.
    \item The customer has accepted the goods or services.\\
\end{itemize}

%-------------------------------------------------------------------------
\section{Recognising Expenses}

Expenses should be matched to the revenue they helped to generate. The expenses associated with an item of revenue must be taken into account in the same reporting period as that in which the item of revenue is included.

%-------------------------------------------------------------------------
\section{Profit, Cash, and Accruals Accounting}

For a particular reporting period, total revenue is not the same as total cash received, and total expenses are not the same as total cash paid. As a result, the profit for the period will not normally represent the net cash generated during that period. This reflects the difference between profit and liquidity.

\begin{blackbox}
    \textbf{Accruals Convention:}\\
    The accruals convention asserts profit is the excess of revenue over expenses for a period, not the excess of cash receipts over cash payments.\\
    The approach to accounting that is based on this convention is frequently referred to as accruals accounting. Both the statement of financial position and the income statement are both prepared on the basis of accruals accounting.
\end{blackbox}

%-------------------------------------------------------------------------
\section{Depreciation}

Depreciation is an attempt to measure the portion of the cost of a non-current asset that has been depleted in generating the revenue recognised during a particular period. In the case of intangibles, the expense is usually referred to as amortisation.\\

A non-current asset has both a physical life and an economic life. The physical life will be exhausted through the effects of wear and tear, etc. The economic life is decided by the effects of technological progress, changes in demand, or changes in the way the business operates.\\

The benefits provided by an asset are eventually outweighed by the costs. The economic life determines the expected \textit{useful life} of an asset for depreciation purposes. \\

\begin{blackbox}
    \textbf{Residual Value:}\\
    When a business disposes of an asset that may still be of value to others, a payment may be received. The payment will represent the residual value, or disposal value, of the asset.\\

    To calculate the total amount to be depreciated, the residual value must be deducted from the cost of the asset.
\end{blackbox}

%-------------------------------------------------------------------------
\section{Costing Inventories}

To calculate the cost of inventories, an assumption must be made about the physical flow of inventories through the business. This assumption must be made about the physical flow of inventories through the business. The assumption need not have anything to do with how inventories actually flow through the business, it is only concerned with providing useful measures of performance and position. Common assumptions used are;
\begin{itemize}
    \item \textbf{First In, First Out (FIFO)} - inventories are costed as if the earliest acquired inventories held are the first to be used. 
    \item \textbf{Last In, First Out (LIFO)} - inventories are costed as if the latest acquired inventories are the first to be used.
    \item \textbf{Weighted Average Cost (AVCO)} - inventories are costed as if they lose their separate identity and go into a 'pool'. Any issues of inventories from this pool will reflect the weighted average cost of inventories held.
\end{itemize}

%-------------------------------------------------------------------------
\section{Trade Receivables Problems}

There is always the risk that the customer will not pay the amount due. When it becomes reasonably certain that the customer will not pay, the amount owed is considered to be a \textbf{bad debt}, which must be taken into account when preparing financial statements.\\

The bad debt must then be 'written off'. This will involve reducing the trade receivables and increasing expenses (by creating an expense known as 'bad debts written off') by the amount of the bad debt.\\

A business must try to determine the about of trade receivables that at the end of the period are doubtful. Once a figure has been derived, an expense known as an allowance for trade receivables should be recognised, which will be shown as an expense and deducted from the total trade receivables figure.

%-------------------------------------------------------------------------
\chapter{Accounting for Limited Companies}
%-------------------------------------------------------------------------
\begin{blackbox}
    \textit{\textbf{Learning Outcomes:}}
    \begin{itemize}
        \item Discuss the nature and financing of a limited company.
        \item Describe the main features of the equity in a limited company and the restrictions placed on owners withdrawing this.
        \item Explain how the financial statements differ in detail in limited companies compared to sole proprietorships and partnerships.
        \item Discuss the reasons for the formation of groups of companies and the main features of financial statements prepared for groups of companies.
    \end{itemize}
\end{blackbox}

%-------------------------------------------------------------------------
\section{The Main Features of Limited Companies}

\subsection{Legal Nature}
A limited company has many rights and obligations people have. It can enter contracts in its own name, sue other people and corporations, and be sued. This contrasts sharply with unincorporated businesses.\\

With the rare exceptions of those created by Act of Parliament or Royal Charter, all companies must be incorporated by registration. The founders, called 'promoters' must fill in forms, and then the Registrar of Companies will enter the new company into the Registry of Companies.\\

The legal separateness of the limited company and its shareholders leads to two important features of the limited company - perpetual life and limited liability.

\subsection{Perpetual Life}
A company is normally granted a perpetual existence and so will continue even when owners of shares pass away. Although, it is possible for either the shareholders or the courts to bring the existence of the company to an end. Shareholders may agree to end the company when it has achieved the purpose for which it was formed or when they feel the company has no real future. Courts may bring an end to the company when creditors have applied to the courts for this to be done because they have not been paid. When shareholders agree, it is a 'voluntary liquidation'.

\subsection{Limited Liability}
Since the company is a legal person in its own right, it must take responsibility for its own debts and losses. Therefore shareholders can limit their losses to the amount they have paid or agreed to pay for their shares.

\subsection{Legal Safeguards}
Various safeguards exist to protect individuals and businesses contemplating dealing with a limited company. The limited liability status must be included in the name of the company, alerting suppliers and lenders to potential risks involved. A further safeguard is the restrictions replaced on the ability of shareholders to withdraw their equity from the company.\\

Limited companies also must produce annual financial statements, and make these publicly available.

\subsection{Private and Public Companies}
A company must be registered as either a public or private company, with the main difference being a public limited company can offer its shares to the general public, but a private limited company cannot. A public limited company must include 'plc' in its name, and a private limited company must include 'Ltd'.

\subsection{Taxation}
Companies are charged \textbf{corporation tax} on their profits and gains. It is levied on the company's taxable profit, which may differ from the profit on the income statement. Generally, they tend to be close to one another. With sole proprietorships and partnerships, tax is levied not on the business but the owner, so will not be reporte din the financial statements of unincorporated businesses.

\subsection{The Stock Exchange}
A \textbf{Stock Exchange} exists as both an important \textit{primary} and \textit{secondary} capital market for companies. As a primary market, its function is to enable investors to sell their securities with ease. As with most stock exchanges, only the shares of listed companies may be traded on the \textbf{London Stock Exchange}.

%-------------------------------------------------------------------------
\section{Managing a Company}

The most senior level of management of a company is the board of directors. \textbf{Directors} are elected by shareholders to manage the company on a day-to-day basis on their behalf. By law, there must be at least one director for an Ltd and two for a plc.

%-------------------------------------------------------------------------
\section{Financing Limited Companies}

\subsection{Equity}
The equity is divided between shares in limited companies - the original investment, and reserves (profit and gains subsequently made). There is also the possibility that there will be more than one type of shares and reserves, so within the basic divisions of share capital and reserves, there will also be subdivisions.\\

When a company is formed, those who take steps to form it will decide how much needs to be raised from potential shareholders to set the company up with the necessary assets to operate. Profit is known as a \textbf{revenue reserve} as it arises from generating revenue. This is not merged with the share capital, as this must be kept separate in order to satisfy company law. 

\subsection{Ordinary Shares}
\textbf{Ordinary shares} represent the basic units of ownership of a business. They are issued by all companies and are often known as \textbf{equities}. Ordinary shareholders are the primary risk-takers as they share in the profits of the company only after all other claims have been satisfied. The potential rewards available reflect the risk they are prepared to take. Usually, only ordinary shareholders are able to vote on issues that affect the company, like the appointment of directors.

\subsection{Preference Shares}
Some companies issue \textbf{preference shares}, which guarantee that if a dividend is paid, the preference shareholders will be entitled to the first part of it up to a maximum value. This value is normally defined as a fixed percentage of the nominal value of the preference shares.

\subsection{Reserves}
The shareholder's equity consists of share capital and reserves. A reason past profits and gains may no longer form part of the shareholder's equity is that they have been paid out to shareholders, eg as dividends. Another reason is the reserves will be reduced by the amount of any losses that the company might suffer. In the same way profits increase equity, losses reduce it.\\

Retained earnings represent overwhelmingly the largest source of new finance for UK companies.\\

Capital reserves arise when issuing shares above their nominal value, and when revaluing non-current assets.

\subsection{Bonus Shares}
It is an option for a company to take reserves of any kind and turn them into share capital. This will involve transferring the desired amount from the reserve concerned to share capital and then distributing the appropriate number of new shares to the existing shareholders - known as \textbf{bonus shares}.

\subsection{Terms}
\begin{itemize}
    \item \textbf{Issued share capital} is share capital that has been issued to shareholders.
    \item \textbf{Paid-up share capital} is the amount the shareholder has already paid for shares.
    \item \textbf{Called-up share capital} is the remaining amount owed to pay for shares. 
    \item \textbf{Fully paid shares} are when the shareholder has paid for them in full.
\end{itemize}

%-------------------------------------------------------------------------
\section{Borrowings}

Many companies borrow in a way that individual investors are able to lend only part of the total amount required. This is particularly the case with the larger, Stock Exchange listed companies and involves them making an issue of \textbf{loan notes.} Although such an issue may be large, private individuals and investing institutions can take it up in small slices. In some cases, these slices of loans can be bought and sold through the Stock Exchange. This means investors do not have to wait for the full term of their loan to obtain repayment. They can sell their slice of the loan to another would-be lender at intermediate points during the loan term.\\

Loan notes are often known as loan stock, bonds, or debentures.

%-------------------------------------------------------------------------
\section{Raising Share Capital}

After an initial share issue, a company may decide to make further issues of new shares in order to finance its operations. These new share issues may be carried out in various ways. They may not involve direct appeals to investors or may employ the services of financial intermediaries. The most common methods of share issue are:
\begin{itemize}
    \item \textbf{Rights issues} - issues made to existing shareholders, in proportion to their existing shareholding.
    \item \textbf{Public issues} - issues made to the general investing public.
    \item \textbf{Private placings} - issues made to selected individuals who are usually approached and asked if they would be interested in taking up new shares.
\end{itemize}

%-------------------------------------------------------------------------
\section{Withdrawing Equity}

Paying dividends is the most usual way of enabling shareholders to withdraw part of their equity. An alternative is for the company to buy its own shares from those shareholders wishing to sell them - a \textbf{share repurchase}. The total revenue reserves appearing on the statement of financial position is rarely the total of all trading profits since the company was first formed as it will have been reduced by either; corporation tax, any dividend paid or amounts paid to the purchase the company's own shares, or any losses from trading and the disposal of non-current assets.\\

The non-withdrawable part consists of share capital plus profits arising from shareholders buying shares in the company and from upward revaluations of assets still held. The law does not specify the size of the non-withdrawable part of shareholders' equity, however for a company to gain the confidence of prospective lenders and suppliers, the bigger the better.

%-------------------------------------------------------------------------
\section{Accounting for Groups of Companies}

Most large businesses operate not as a single company but as a group of companies. One company (the \textbf{parent company} or \textbf{holding company}) is able to control various subsidiary companies, normally as a result of owning more than 50 per cent of their shares. The reasons why many businesses operate in groups include;
\begin{itemize}
    \item A desire for each part of the business to have its own limited liability, so financial problems in one part of a business cannot have an adverse effect on other parts.
    \item An attempt to make each part of the business have some sense of independence and autonomy.\\
\end{itemize}

Each company within a group will prepare its own individual annual financial statements, and the parent company will prepare \textbf{consolidated financial statements}.

\begin{itemize}
    \item \textbf{Goodwill arising on consolidation} - occurs when a parent acquires a subsidiary and pays more than the subsidiary's assets appear to be worth.\\
    This could represent the value of a good representation, or a loyal and skilled workforce.\\
    Goodwill arising on consolidation will appear as an intangible non-current asset.
    \item \textbf{Non-controlling interests (NCI)} - a principle followed when preparing group statements is all of the revenue, expenses, assets, liabilities and cash flows of each subsidiary are reflected to their full extent in the group financial statements. This is true whether or not the parent owns all of the shares or not, provided they have control. Where not all the shares are owned by the parent, the investment of those shareholders in the subsidiary appear as part of the shareholder's equity in the group statement of financial position.\\
    The statement will reflect the fact not all the profit of the group is attributable to the shareholders of the parent company - a part of it goes to 'outside' shareholders.\\
    The interests of outside shareholders are known as \textbf{non-controlling interests}.
\end{itemize}

%-------------------------------------------------------------------------
\chapter{The Statement of Cash Flows}
%-------------------------------------------------------------------------
\begin{blackbox}
    \textit{\textbf{Learning Outcomes:}}
    \begin{itemize}
        \item Discuss the crucial importance of cash to a business.
        \item Explain the nature of the statement of cash flows and discuss how it can be helpful in identifying cash flow problems.
        \item Prepare a statement of cash flows.
        \item Interpret a statement of cash flows.
    \end{itemize}
\end{blackbox}

%-------------------------------------------------------------------------
\section{Why Cash is Important}

In one sense, cash is just another asset that the business needs to enable it to function. But the importance of cash lies in the fact people will only normally accept cash in settlement of their claims. If a business wants to employ people, it must pay them in cash. When businesses fail, it is the lack of cash to pay amounts owed that really pushes them under. Cash generation is vital for businesses to survive and be able to take advantage of commercial opportunities, making cash the preeminent business asset.\\

%-------------------------------------------------------------------------
\section{The Statement of Cash Flows}

\begin{blackbox}
    \textbf{Statement of Cash Flows:}\\
    The Statement of Cash Flow summarises the inflows and outflows of cash (and cash equivalents) for a business over a period. To aid user understanding, these cash flows are divided into categories (eg those relating to investments in non-current assets). Cash inflows and outflows falling within each category are reported on the statement of cash flows. By adding the totals for each category together, an overall total is achieved that reveals the net increase or decrease in cash over the period.
\end{blackbox}

The statement follows the requirements of International Accountant Standard (\textbf{IAS}) \textit{7 Statement of Cash Flows}, which applies to Stock Exchange listed companies.\\

\subsection{Cash Flows from Operating Activities}
These are the cash inflows and outflows from normal day-to-day trading after taking into account of the tax paid and financing costs relating to it. The cash inflows are amounts received from trade receivables (credit customers settling their accounts) and from cash sales for the period. The cash outflows are the amounts paid for inventories, operating expenses, corporation tax, interest and dividends.

\subsection{Cash Flows from Investing Activities}
These include cash outflows to acquire non-current assets, and cash inflows from their disposal. Non-current assets may include financial instruments made in loans or shares in another business. 

\subsection{Cash Flows from Financing Activities}
These represent cash flows relating to the long-term financing of the business. Under IAS 7, interest and dividends paid by the business could appear here as outflows. They could also be included in \textit{operating activities} as they represent a cost of raising finance.

\subsection{Net Increase or Decrease in Cash and Cash Equivalents}
The final total shown on the statement will be the net increase or decrease in cash and cash equivalents over the period. It will be deducted from the totals from each of the three categories mentioned.

%-------------------------------------------------------------------------
\section{Preparing the Statement of Cash Flows}

\subsection{Deducing Net Cash Flows from Operating Activities}
The direct method involves an analysis of the cash records of the business for the period, identifying all payments and receipts relating to operating activities. These receipts and payments are then summarised to provide the total figures for inclusion in the statement of cash flows.\\

A much more popular approach is the indirect method. This relies on that sooner or later sales revenue gives rise to cash inflows and expenses give rise to outflows. This means that the figure for profit for the year will be linked to the net cash flows from operating activities. Since businesses have to produce an income statement, the information it contains can be used as a starting point.

\subsection{Deducing the Other Areas}
Deriving cash flows from investing activities and the cash flows from financing activities largely involves a comparison of the opening and closing statements of financial position to detect movements of non-current assets and liabilities, and equity over the period.

%-------------------------------------------------------------------------
\section{Reconciliation of Liabilities from Financing Activities}

\textbf{IAS 7} requires businesses to provide a reconciliation that shows the link between liabilities at the beginning and end of the reporting period that relate to cash flows from financing activities in the statement of cash flows. This reconciliation sets out movements in liabilities, such as long-term borrowings and lease liabilities, over the reporting period. A separate reconciliation is required for each type of liability. The reconciliation appears as a note to the statement of cash flows and is designed to help track any changes occurring in the liabilities of the business.

%-------------------------------------------------------------------------
\chapter{Analysing and Interpreting Financial Statements}
%-------------------------------------------------------------------------
\begin{blackbox}
    \textbf{\textit{Learning Outcomes:}}
    \begin{itemize}
        \item Identify the major categories of ratios that can be used for analysing financial statements.
        \item Calculate key ratios for assessing the financial performance and position of a business and explain their significance.
        \item Discuss the use of ratios in helping predict financial failure, and their limitations as a tool of financial analysis.
    \end{itemize}
\end{blackbox}

%-------------------------------------------------------------------------
\section{Financial Ratio Classifications}

\begin{blackbox}
    \textbf{Profitability:}\\
    Profitability ratios express the profit made in relation to other key figures in the financial statements or to some business resource.
\end{blackbox}

\begin{blackbox}
    \textbf{Efficiency:}\\
    Ratios may be used to measure the efficiency resources (eg inventories or employees) have been used within the business. These ratios are also referred to as activity ratios.
\end{blackbox}

\begin{blackbox}
    \textbf{Liquidity:}\\
    It is vital there are sufficient liquid resources available to meet maturing obligations. Liquidity ratios examine the relationship between liquid resources and amounts due for payment in the near future.
\end{blackbox}

\begin{blackbox}
    \textbf{Financial Gearing:}\\
    This is the relationship between the contribution to financing a business made by the owners and the contribution made by others in the form of loans. Gearing ratios help to reveal the extent to which loan finance is used and the effect on the level of risk borne by a business.
\end{blackbox}

\begin{blackbox}
    \textbf{Investment:}\\
    Ratios concerned with assessing the returns and performances of shares in a business from the perspective of shareholders.
\end{blackbox}

%-------------------------------------------------------------------------
\section{Comparisons}

\subsection{Past Periods}
Comparing the ratio calculated with the same one for a previous period, an improvement or deterioration in performance can be detected. There are problems however, as there is the possibility trading conditions were quite different, and when comparing a single business over time, operating inefficiencies may not be noticed.

\subsection{Similar Businesses}
A useful basis for comparing a particular ratio is the ratio achieved by a similar business in the same period. This too has problems, such as competitors may have different accounting policies, different year-ends, and it could be difficult to obtain financial information from sole proprietors as it does not legally have to be made available.

\subsection{Planned Performance}
Ratios based on actual results may be compared with targets that management developed, based on realistic assumptions. Target ratios may be prepared for each aspect of the business' activities, taking into account the past performance and performance of other businesses.

%-------------------------------------------------------------------------
\section{Profitability}

\begin{blackbox}
    \textbf{Return on Ordinary Shareholder's Funds:}\\
    This compares the amount of profit for the period available to owners with their average investment in the business during that same period.
    \begin{equation}
        ROSF = \frac{Profit \: for \: the \: year}{Ordinary \: share \: capital + Reserves}
    \end{equation}
\end{blackbox}

\begin{blackbox}
    \textbf{Return on Capital Employed:}\\
    This is a fundamental measure of business performance, expressing the relationship between the operating profit generated and the average long-term capital invested in the business.
    \begin{equation}
        ROCE = \frac{Operating \: profit}{Share \: capital +  Reserves + Non-current \: liabilities}
    \end{equation}
    ROCE is considered by many to be the primary measure of profitability, as it compares capital invested with operating profit, to reveal the effectiveness with which funds have been deployed.
\end{blackbox}

\begin{blackbox}
    \textbf{Operating Profit Margin:}\\
    This relates the operating profit for the period to the sales revenue.
    \begin{equation}
        Operating \: profit \: margin = \frac{Operating \: profit}{Sales \: revenue}
    \end{equation}
\end{blackbox}

\begin{blackbox}
    \textbf{Gross Profit Margin:}\\
    This relates the gross profit to the sales revenue.
    \begin{equation}
        Gross \: profit \: margin = \frac{Gross \: profit}{Sales \: revenue}
    \end{equation}
\end{blackbox}

%-------------------------------------------------------------------------
\section{Efficiency}

\begin{blackbox}
    \textbf{Average Inventories Turnover Period:}\\
    Inventories often represent a significant investment for a business, and may account for a substantial proportion of total assets held. This ratio measures the average period that inventories are held.
    \begin{equation}
        Average \: inventories \: turover \: period = \frac{Average \: inventories \: held}{Cost \: of \: sales} \cdot 365
    \end{equation}
\end{blackbox}

\begin{blackbox}
    \textbf{Average Settlement Period for Trade Receivables:}\\
    Other than retailers, selling on credit is the norm, so trade receivables are common. A business will want to keep the funds tied up in trade receivables to a minimum, and the speed of payment can have a significant impact on the business' cash flow. This ratio indicates how long, on average, credit customers take to pay the amounts they owe to the business.
    \begin{equation}
        Average \: settlement \: period \: for \: TR = \frac{Average \: trade \: receivables}{Credit \: sales \: revenue} \cdot 365
    \end{equation}
\end{blackbox}

\begin{blackbox}
    \textbf{Average Settlement Period for Trade Payables:}\\
    This measures how long, on average, the business takes to pay those who have supplied goods and services on credit.
    \begin{equation}
        Average \: settlement \: period \: for \: TP = \frac{Average \: trade \: payables}{Credit \: purchases} \cdot 365
    \end{equation}
\end{blackbox}

\begin{blackbox}
    \textbf{Sales Revenue to Capital Employed:}\\
    This examines how effectively the assets of the business are being used to generate sales revenue.
    \begin{equation}
        Sales \: revenue \: to \: capital \:employed \: ratio = \frac{Sales \: revenue}{Share \: capital + Reserves + NC \: Liabilities}
    \end{equation}
\end{blackbox}

\begin{blackbox}
    \textbf{Sales Revenue per Employee:}\\
    This ratio relates sales revenue generated to labour, providing a measure of productivity in the workforce.
    \begin{equation}
        Sales \: revenue \: per \: employee = \frac{Sales \: revenue}{Number \: of \: employees}
    \end{equation}
\end{blackbox}

%-------------------------------------------------------------------------
\section{Liquidity}

\begin{blackbox}
    \textbf{Current Ratio:}\\
    This ratio compares the liquid assets of the business with current liabilities.
    \begin{equation}
        Current \: ratio = \frac{Current \: assets}{Current \: liabilities}
    \end{equation}
    The higher the current ratio, the more liquid a business is considered to be. However, this is not always good as if a business has an very high ratio, excessive funds are tied up in cash and may not be used as productively as they otherwise could be.
\end{blackbox}

\begin{blackbox}
    \textbf{Acid Test Ratio:}\\
    This is similar to the current ratio, nut is a more stringent test of liquidity.
    \begin{equation}
        Acid \: test \: ratio = \frac{Current \: assets \: (excluding \: inventories)}{Current \: liabilities}
    \end{equation}
\end{blackbox}

\begin{blackbox}
    \textbf{Cash Generated from Operations to Maturing Obligations Ratio:}\\
    This ratio compares the cash generated from operations with the current liabilities of the business, providing an indication of the ability of the business to meet its maturing obligations.
    \begin{equation}
        Cash \: from \: operations \: to \: maturing \: obligations \: ratio = \frac{Cash \: from \: obligations}{Current \: liabilities}
    \end{equation}
\end{blackbox}

%-------------------------------------------------------------------------
\section{Financial Gearing (Leveraging)}

\begin{blackbox}
    \textbf{The Gearing Ratio:}\\
    This measures the contribution of long-term lenders to the long-term capital structure of a business.
    \begin{equation}
        Gearing \: ratio = \frac{Long-term \: liabilities}{Share \: capital + Reserves + \: Long-term \: liabilities}
    \end{equation}
\end{blackbox}

\begin{blackbox}
    \textbf{Interest Cover Ratio:}\\
    This ratio measures the amount of operating profit available to cover interest payable.
    \begin{equation}
        Interest \: cover \: ratio = \frac{Operating \: profit}{Interest \: payable}
    \end{equation}
\end{blackbox}

%-------------------------------------------------------------------------
\section{Investments}

\begin{blackbox}
    \textbf{Dividend Payout Ratio:}\\
    A ratio that measures the proportion of earnings paid out to shareholders in the form of dividends.
    \begin{equation}
        Dividend \: payout \: ratio = \frac{Dividends \: announced \: for \: the \: year}{Earnings \: for \: the \: year \: available \: for \: dividends}
    \end{equation}
\end{blackbox}

\begin{blackbox}
    \textbf{Dividend Cover Ratio:}
    \begin{equation}
        Dividend \: cover \: ratio = \frac{Earnings \: for \: the \: year \: available \: for \: dividend}{Dividends \: announced \: for \: the \: year}
    \end{equation}
\end{blackbox}

\begin{blackbox}
    \textbf{Dividend Yield Ratio:}\\
    This relates to the cash return from a share to its current market value.
    \begin{equation}
        Divident \: yield = \frac{Dividend \: per \: share}{Market \: value \: per \: share}
    \end{equation}
\end{blackbox}

\begin{blackbox}
    \textbf{Earnings per Share}:\\
    This ratio relates the earnings generated by the business available to shareholders to the number of shares in issue. 
    \begin{equation}
        EPS = \frac{Earnings \: available \: to \: ordinary \: shareholders}{Number \: of \: ordinary \: shares \: in \: issue}
    \end{equation}
\end{blackbox}

\begin{blackbox}
    \textbf{Cash Generated from Operations per Share:}\\
    The cash generated from operations can provide a good guide to the ability of a business to pay dividends and to under-take planned expenditures.
    \begin{equation}
        Cash \: from \: operations \: per \: share = \frac{Cash \: from \: operations \: (less \: preference \: dividend)}{Number \: of \: ordinary \: shares \: in \: issue}
    \end{equation}
\end{blackbox}

\begin{blackbox}
    \textbf{Price/Earnings Ratio:}\\
    This ratio relates the market value of a share to the earnings per share.
    \begin{equation}
        P/E \: ratio = \frac{Market \: value \: per \: share}{Earnings \: per \: share}
    \end{equation}
\end{blackbox}

%-------------------------------------------------------------------------
\section{The Problems of Overtrading}

\textbf{Overtrading} occurs when a business is operating at a level of activity that cannot be supported by the amount of finance that has been committed. This situation is often due to poor financial control over the business by its managers. The underlying reasons for overtrading are varied;
\begin{itemize}
    \item Young, expanding businesses that fail to adequately prepare for rapid increases in demand for their goods or services, leading to insufficient finance to fund the level of inventories needed to support the level of sales.
    \item Businesses where the manager has misjudged the level of expected sales and, underestimated the level of working capital required or failed to control escalating costs.
    \item As a result of inflation, causing more finance to have to be committed.
    \item When the owners are unable to inject further funds themselves and they cannot persuade others to invest.
\end{itemize}

%-------------------------------------------------------------------------
\section{Using Ratios to Predict Financial Failure}

\subsection{Single Ratios}
Various ways of using ratios to predict future financial failure has been developed. Early research tracked a particular ratio to see whether it was a good or bad predictor of financial failure.\\

It was found that businesses that ultimately went on to fail were characterised by lower rates of return, higher levels of gearing, lower levels of coverage for their fixed interest payments, and more variable returns on shares.\\

This approach is referred to as \textbf{univariate analysis} because it looks at one ratio at a time.

\subsection{Combinations of Ratios}
Weakness of univariate analysis has led researchers to develop models comining ratios to produce a single index that can be interpreted more clearly. One approach employs \textbf{multiple discriminate analysis (MDA)} which is a statistical technique similar to regression analysis and which can be used to draw a boundary between those businesses that fail and those businesses that do not - the \textbf{discriminate function}.
%-------------------------------------------------------------------------

\section{Limitations of Ratio Analysis}

\subsection{Quality of Financial Statements}
Since ratios are based on financial statements, they will inherit the limitations of these statements. Internally generated goodwill are excluded from the statement of financial position, so even though they may be of considerable value, they will not be included in ratios such as ROSF and ROCE.

\subsection{Inflation}
Inflation may distort financial results, for example the reported value of assets will be understated in current terms during a period of inflation. Another example is inventories may be acquired several months before they are sold, and during a period of inflation, this will mean that the expense may not reflect prices current at the time of sale.

\subsection{The Restricted View Given by Ratios}
Ratios can not be exclusively relied on, as while one business may have a higher ROCE than another business, but the other business may generate much more absolute profit, which is still very important.

%-------------------------------------------------------------------------
\chapter{Summary}
%-------------------------------------------------------------------------
\section{Chapter 1 - Introduction to Accounting}

\textbf{Providing a service}
\begin{itemize}
    \item Accounting can be viewed as a form of service as it involves providing financial information to users.
    \item Fundamental qualities required are relevance and faithful representation. Other qualities that enhance the usefulness are comparability, verifiablity, timeliness, and understandability.
    \item Providing a service can be costly, so in theory, financial information should be produced only if the cost is less than the benefits gained.\\
\end{itemize}

\textbf{Accounting information}
\begin{itemize}
    \item Accounting is part of the total information system within a business. It shares features common to all information systems - identification, recording, analysis, reporting of information.\\
\end{itemize}

\textbf{Management accounting and financial accounting}
\begin{itemize}
    \item Management accounting seeks to meet the needs of the business' managers, financial accounting seeks to meet the needs of providers.
    \item The two stands differ in the types of reports produced, level of reporting detail, time orientation, regulation, and range and quality of information provided.\\
\end{itemize}

\textbf{Types of businesses}
\begin{itemize}
    \item Sole proprietorship - easy to set up and flexible to operate but the owner has unlimited liability.
    \item Partnership - easy to set up and spreads burden of ownership, but partners tend to have unlimited liability and there are ownership risks if the partners are unsuitable, etc.
    \item Limited company - limited liability for owners but obligations imposed on the way a company conducts its affairs.\\
\end{itemize}

\textbf{Wealth creation}
\begin{itemize}
    \item The key financial objective of a business is to enhance the wealth of the owners.
    \item However needs of employees, suppliers, and the local community must not be ignored.
    \item When setting financial objectives, the right balance must be struck between risk and return.\\
\end{itemize}

%-------------------------------------------------------------------------
\section{Chapter 2 - Statement of Financial Position}

\textbf{The statement of financial position}
\begin{itemize}
    \item This sets out the assets of businesses on the one hand, and the claims against those assets on the other.
    \item Assets are resources of the business, with certain characteristics like the right to potential economic benefits.
    \item Claims are the obligations on the part of the business to provide cash, or some other benefit to outside parties.
    \item Claims are either equity - the claims of the owners, or liabilities - the claims of others.
    \item $Assets = Equity + Liabilities$\\
\end{itemize}

\textbf{Classification of assets and liabilities}
\begin{itemize}
    \item Current assets are cash or near cash, or are held for sale or consumption in the normal course of business.
    \item Non-current assets are normally held for long-term operations of the business.
    \item Current liabilities represent amounts due in the normal course of the business' operating cycle, held for trading, or are to be settled within a year of the reporting period.
    \item Non-current liabilities represent amounts due that are not current liabilities.\\
\end{itemize}

\textbf{Conventions}
\begin{itemize}
    \item The main conventions relating to the statement of financial position are;
    \begin{itemize}
        \item business entity convention
        \item historic cost convention
        \item prudence convention
        \item going concern convention
        \item dual aspect convention\\
    \end{itemize}
\end{itemize}

\textbf{Asset valuation}
\begin{itemize}
    \item The initial treatment is to show non-current assets at historic cost.
    \item Fair values may be obtained instead, provided this can be done reliably.
    \item Non-current assets with finite useful lives should be shown at cost less any amortisation.
    \item Inventories are shown at the lower cost or net realisable value.
\end{itemize}

%-------------------------------------------------------------------------
\section{Chapter 3 - The Income Statement}

\textbf{The income statement}
\begin{itemize}
    \item The income statement reveals how much profit or loss has been generated over a period.
    \item Gross profit is the sales revenue less the cost of sales.
    \item Operating profit is gross profits less overheads.
    \item Profit for the period is operating profit plus non-operating income less non-operating costs.\\
\end{itemize}

\textbf{Expenses and revenue}
\begin{itemize}
    \item Cost of sales can be identified by matching the cost of each sale to the particular sale or adjusting the goods bought during a period by the opening and closing inventories.
    \item Revenue is recognised when a business has performed its obligations - when control of the good or service is passed to the customer.
    \item The matching convention states expenses should be matched to the revenue that they help generate.
    \item An expense reported in the income statement may not be the same as the cash paid. This can result in accruals or prepayments appearing in the statement of financial position.
    \item The materiality convention states that all important and relevant facts should be included in the statement. 
    \item The accruals convention is an accounting system that tries to match the recognition of revenues earned with the expenses incurred in generating those revenues. It ignores the timing of the cash flows associated with revenues and expenses.\\
\end{itemize}

\textbf{Depreciation of non-current assets}
\begin{itemize}
    \item Depreciation requires consideration of the cost (or fair value), useful life, and residual value of an asset.
    \item The straight-line method allocates the amount to be depreciated evenly over the useful life of an asset.
    \item The reducing-balance method applies a fixed percentage rate of depreciation to the carrying amount of an asset each year.\\
\end{itemize}

\textbf{Costing inventories}
\begin{itemize}
    \item FIFO - earliest inventories held are first to be used.
    \item LIFO - latest inventories held are first to be used.
    \item AVCO - applies an average cost to all inventories used.
\end{itemize}

%-------------------------------------------------------------------------
\section{Chapter 4 - Accounting for Limited Companies}

\textbf{Main features of a limited company}
\begin{itemize}
    \item It has a separate life to its owners and is granted a perpetual existence.
    \item It must take responsibility for its own debts and losses.
    \item The limited liability status is included as part of the business name, restrictions are placed on the ability of owners to withdraw equity and annual financial statements are made publicly available.
    \item A company can offer its shares for sale to the public - private company cannot.
    \item A limited company is governed by a board of directors elected by the shareholders.\\
\end{itemize}

\textbf{Financing a limited company}
\begin{itemize}
    \item The share capital of a company can be of two main types, ordinary shares and preference shares.
    \item Ordinary shares are the main risk takers and are given voting rights.
    \item Preference shares a given a right to a fixed dividend before ordinary shareholders receive a dividend.
    \item Reserves are profits and gains made by the company and form part of the ordinary shareholder's claim.
    \item Borrowings provide another major source of finance.\\
\end{itemize}

\textbf{Share issues}
\begin{itemize}
    \item Bonus shares are issued to existing shareholders when part of the reserves of the company is converted into share capital. No funds are raised.
    \item Shares may be issued for cash through a rights issue, a public issue, or a public placing.
    \item The shares of a company may be traded on the London Stock Exchange.\\
\end{itemize}

\textbf{Withdrawing equity}
\begin{itemize}
    \item Revenue reserves arise from trading profits and from realised profits on the sale of NC assets.
    \item Capital reserves arise from the issue of shares above their nominal value or from upward revaluation of NC assets.
    \item Revenue reserves can be withdrawn as dividends by the shareholders whereas capital reserves normally cannot.\\
\end{itemize}

\textbf{Financial statements of limited companies}
\begin{itemize}
    \item The income statement has two measures of profit displayed after the operating profit - profit before tax and profit for the year.
    \item The income statement also shows audit fees and tax on profits for the year.
    \item The statement of financial position will show any unpaid tax and any unpaid, but authorised, dividends as current liabilities.
    \item The share capital plus the reserves make up 'equity'.
\end{itemize}

%-------------------------------------------------------------------------
\section{Chapter 5 - The Statement of Cash Flows}

\textbf{The need for a statement of cash flows}
\begin{itemize}
    \item Cash is important because no business can operate without it.
    \item The statement reveals cash movements over a period.
    \item Profit for the period is generally not equal to cash generated for that period.\\
\end{itemize}

\textbf{Preparing the statement of cash flows}
\begin{itemize}
    \item There are three major categories of cash flows;
    \begin{itemize}
        \item cash flows from operating activities
        \item cash flows from investing activities
        \item cash flows from financing activities
    \end{itemize}
    \item The total of cash movements will provide the net increase or decrease in cash and cash equivalents for the period.
    \item A reconciliation can be undertaken to check the opening balance of cash and cash equivalents plus the net increase for the period equals the closing balance.\\
\end{itemize}

\textbf{Calculating the cash generated from operations}
\begin{itemize}
    \item The direct method is based on an analysis of the cash records for the period.
    \item The indirect method takes the operating profit for the period, adds back any depreciation change and then adjusts for changes in inventories, receivables and payable during the period.\\
\end{itemize}

\textbf{Reconciliation of liabilities from financing activities}
\begin{itemize}
    \item IAS 7 requires businesses provide a reconciliation that shows the link between liabilities at the beginning and end of the reporting period that relate to cash flows from financing activities.
    \item The reconciliation shows how borrowings and other non-equity finance have changed over the reporting period. Each type of non-equity finance needs to be separately reconciled.
\end{itemize}

%-------------------------------------------------------------------------
\section{Chapter 6 - Analysing and Interpreting Financial Statements}

\textbf{Ratio analysis}
\begin{itemize}
    \item Compares two related figures, usually from the same set of financial statements.
    \item Aids in understanding what financial statements mean
    \item Usually requires the performance for past periods, similar businesses, or planned performance as benchmark ratios.\\
\end{itemize}

\textbf{Profitability ratios}
\begin{itemize}
    \item Concerned with effectiveness at generating profit
    \item Most commonly found in practice are the return on ordinary shareholders' funds (ROSF), the return on capital employed (ROCE), operating profit margin, and gross profit margin.\\
\end{itemize}

\textbf{Efficiency ratios}
\begin{itemize}
    \item Are concerned with efficiency of using assets/resources.
    \item Most commonly found are average inventories turnover, average settlement period, sales revenue to capital employed, sales revenue per employee.\\
\end{itemize}

\textbf{Liquidity ratios}
\begin{itemize}
    \item Are concerned with the ability to meet short-term obligations
    \item Most commonly found in practice are the current ratio, acid test ratio, and cash generated to maturing obligations ratio.\\
\end{itemize}

\textbf{Uses of ratios}
\begin{itemize}
    \item Ratios can be used to detect trends.
    \item Ratios can be used to detect signs of overtrading.
    \item Ratios can be used to predict financial failure.
    \item Univariate analysis uses a single ratio at a time, where as multiple discriminate analysis combines various ratios within a model.
\end{itemize}

\textbf{Limitations of ratio analysis}
\begin{itemize}
    \item Ratios are only as reliable as the financial statements from which they derive.
    \item Inflation can distort the information.
    \item It can be difficult to find a suitable benchmark.
    \item Some ratios could mislead due to the 'snapshot' nature of the statement of financial position.
\end{itemize}

%-------------------------------------------------------------------------
\chapter{Summary: Equations}

\begin{blackbox}
    \textbf{Return on Ordinary Shareholder's Funds:}
    \begin{equation}
        ROSF = \frac{Profit \: for \: the \: year}{Ordinary \: share \: capital + Reserves}
    \end{equation}
\end{blackbox}

\begin{blackbox}
    \textbf{Return on Capital Employed:}
    \begin{equation}
        ROCE = \frac{Operating \: profit}{Share \: capital +  Reserves + Non-current \: liabilities}
    \end{equation}
\end{blackbox}

\begin{blackbox}
    \textbf{Operating Profit Margin:}
    \begin{equation}
        Operating \: profit \: margin = \frac{Operating \: profit}{Sales \: revenue}
    \end{equation}
\end{blackbox}

\begin{blackbox}
    \textbf{Gross Profit Margin:}
    \begin{equation}
        Gross \: profit \: margin = \frac{Gross \: profit}{Sales \: revenue}
    \end{equation}
\end{blackbox}

\begin{blackbox}
    \textbf{Average Inventories Turnover Period:}
    \begin{equation}
        Average \: inventories \: turover \: period = \frac{Average \: inventories \: held}{Cost \: of \: sales} \cdot 365
    \end{equation}
\end{blackbox}

\begin{blackbox}
    \textbf{Average Settlement Period for Trade Receivables:}
    \begin{equation}
        Average \: settlement \: period \: for \: TR = \frac{Average \: trade \: receivables}{Credit \: sales \: revenue} \cdot 365
    \end{equation}
\end{blackbox}

\begin{blackbox}
    \textbf{Current Ratio:}
    \begin{equation}
        Current \: ratio = \frac{Current \: assets}{Current \: liabilities}
    \end{equation}
\end{blackbox}

\begin{blackbox}
    \textbf{Acid Test Ratio:}
    \begin{equation}
        Acid \: test \: ratio = \frac{Current \: assets \: (excluding \: inventories)}{Current \: liabilities}
    \end{equation}
\end{blackbox}

\begin{blackbox}
    \textbf{Cash Generated from Operations to Maturing Obligations Ratio:}
    \begin{equation}
        Cash \: from \: operations \: to \: maturing \: obligations \: ratio = \frac{Cash \: from \: obligations}{Current \: liabilities}
    \end{equation}
\end{blackbox}

\end{document}







